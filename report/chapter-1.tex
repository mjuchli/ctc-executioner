\chapter{Introduction}

Financial institutions make decisions on whether to buy or sell assets based on various reasons, including: customer requests, fundamental analysis\cite{fundamental-analysis}, technical analysis\cite{technical-analysis}, top-down investing\cite{td-investing}, bottom-up investing\cite{bu-investing} and many more. 
The high-level trading strategies oftentimes define the purpose of their business and how the institution positions itself in the financial markets and, if existent, towards its customers. 
Regardless of the high-level trading strategy that is being applied, the invariable outcome is the decision to buy or sell assets.
This work aims to make a step towards answering the non-trivial question on how to optimize a buy or sell of an asset on a stock exchange with the use of reinforcement learning techniques.
The following sections will elaborate this problem in detail and state the research objectives of this work. 
We then list the contributions made to the scientific community throughout this work, followed by a brief overview of the structure of this report.

\section{Problem Statement}

We are concerned about which are traded at stock exchanges.
Regarding stock exchanges there is little consensus as to when corporate stock was first traded; some argue that the exchange, in the form as we know it today, dates as far back as 1531, when the East Indian Company stock was traded in Antwerp\cite{stock-exchange}.
Modern financial markets such as the London Stock exchange (LSE), the New York Stock Exchange (NYSE) but also the numerous cryptocurrency exchanges which appeared suddenly in the last few year, all rely on the same very same principles as back then.
Every participant is either willing to buy or sell a given amount of an asset to, respectively for, a certain price.

When in the late '90s the regulatories started to let traders reach into the market using electronic communications networks (ECNs), a new era arose \cite{dark-pools}.
High frequency traders (HFT) and algorithmic traders suddenly gained advantage over the traders who placed their orders manually.
As of today up to 95\% of the institutional investors trade trough electronic channels\cite{ecn}.
This certainly reduced the advantages some parties have over others, however, a certain gap still remains.
The average investor without fibre access to the exchange and supporting algorithms will likely take a small initial loss into account when buying or selling assets -- and might not even be aware of it.

\begin{figure}[H]
    \centering
    \makebox[\linewidth]{
        \includegraphics[width=\linewidth]{orderbook-gdax.png}
    }
    \caption{Order book snapshot: https://www.bitfinex.com/t/BTC:USD}
    \label{fit:intro-orderbook}
\end{figure}

Figure \ref{fit:intro-orderbook} shows a snapshot taken at some time $t$ from the trading pair Bitcoin (BTC) versus US dollar (USD) taken at the Bitfinex\footnote{https://www.bitfinex.com} cryptocurrency exchange.
The order book shows two sides, the parties who are willing to buy on the left and the parties who are willing to sell on the right.
The columns indicate the number of buyers and sellers (\textit{count}) who are willing to buy, respectively sell, a certain \textit{amount} for a given \textit{price}.
The column \textit{total} is simply the cumulative sum of the amount, or volume, on each side.
The two sides are separated by the \textit{spread}. 
In this particular case, the current best \textit{bid} price at which someone is willing to buy, is \$14,910.00 and the best ask-price at which someone is willing to sell, is \$14,930.00. 
Therefore, the spread is currently \$20.00 wide.
\\
\\
Suppose we want to buy 1.0 BTC.
In simplified terms, there are essentially two possible ways to do so:
\begin{enumerate}
    \item Buy $i$ shares (1.0) right away for \$14,930.00 from a seller. We submit a \textit{market} order.
    \item State a price for which we are willing to to buy $i$ shares (1.0) at price $p$, for example at \$14'910.00, and wait until someone is willing to sell for this price. We submit a \textit{limit} order.
\end{enumerate}

Both types of orders come with their advantages and disadvantages.
A market order ensures that we will be able to acquire the stated amount of shares immediately for \$14'930.00, provided that no one else is ahead of us or the seller cancels his/her listing. 
In this case we are automatically willing to pay for the next available best price.
However, we do pay a premium compared to the limit order.

With a limit order the exchange guarantees that we will pay \$14'910.00 or less.
However, this comes with the risk that we will never be able to buy if nobody is going to sell at the mentioned price, which will force us to buy for a higher price within a specified time horizon $H$.
Thus, we pay an \textit{opportunity cost}.
Likewise, if a seller realizes that we are about to buy for price $p$ at which is was willing to sell, he then could \textit{cancel} his offer as his incentive is to sell for a higher price.
Perhaps this has been his/her intention to only post an offer but withdrawal as soon as a counter-offer will be listed that is close to his/her price.
This method is known as \textit{quote stuffing}.
\\
\\
Clearly, posting an order is not as trivial as one would think.
Nevertheless, the prices for some financial products, including cryptocurrencies, are very volatile such that it is likely to be able to buy for less than what is offered at the mentioned time $t$.
Thus, placing orders \textit{deeper} in the book and wait could be beneficial.
According to Guo et. al. \cite{guo2013optimal} algorithmic trading is based on two different time scales: \textit{order execution} concerns about optimally slicing big orders into smaller ones in order to minimize the \textit{price impact}, that is, on a daily or weekly basis.
\textit{Order placement} on the other hand concerns about optimally placing orders within ten to hundred seconds. 
In this thesis we are concerned about the latter, which conforms the context of the problem: at which price level $p$ should one attempt to buy or sell $i$ shares of an asset and what are the factors that might interfere with our intention to do so, within a time horizon $H$ of 100 seconds.

\section{Research objectives}

We study the data from a cryptocurrency exchange of our choice and build a framework that allows to simulate and understand the outcome of order placement in this market.
With this knowledge we then try to overcome the exploitation of other participants in the market by building an intelligent trader that follows a placement strategy which aims to execute orders to a favourable price.

The main objective of this research is to develop a machine learning based model of limit-order placements in an order driven crypto currency market that can be utilised to optimize buying and selling of crypto currencies. To achieve this, we firstly conduct several experiments to analyse the behaviour of previously proposed models in a controlled environment by utilising historical order book data with the aim to identify the limitation of these models. We then propose a reinforcement learning implementation and study the results with respect to the aforementioned limitations. Questions to be answered include:

How does trading time horizon and inventory affect execution?
How does market liquidity provided by bid and ask orders affect execution?
Is the performance of reinforcement learning limited to the findings of an empirical model that models the execution probability?
Another focus of this research is the aim to derive patterns from order book data. Particularly, order book events are observed and an attempt is made to detect behavioural patterns of traders participating in the market. Hence the objective is:

How can data, which is derived from a limit order book, be used as features in a reinforcement learning environment in order to contribute to the optimization of order execution?
Lastly, we make an attempt to improve the execution strategy by incorporating the found patterns. Therefore the aim is to answer:

Does pattern recognition of order book data allow to overcome the limitations of order execution proposed by statistical models and other machine learning approaches?


\section{Contributions}

\section{Document structure}

We first provide in Chapter \ref{chap:preliminaries} background information to the reader concerning the components of a stock exchange and the fundamentals of the closely related time series.
We further make the reader familiar with (Deep) Reinforcement Learning.
In Chapter \ref{chap:related-work} we elaborate on the behaviour of order execution followed by approaches of both statistical and machine learning nature.
Chapter \ref{chap:data} explains the process of data collection and its preparation which was done prior its use in the following chapters.
Namely, Chapter \ref{chap:setup} explains the experimental setup of the Reinforcement Learning environments, the agents and the features being processed and used.
In Chapter \ref{chap:analysis} we then analyze the data and proceed execution placement with various techniques, including the reasoning of our findings.
Finally, Chapter \ref{chap:conclusion} formulates a conclusion of our findings and states a future research direction.
